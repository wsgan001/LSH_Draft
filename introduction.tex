Understanding the processes and dynamics of \emph{information diffusion} through networks plays a fundamental role in a variety of domains, such as evaluating the effects of networks in marketing~\cite{domingos:2001,kempe:2003,leskovec:2007a}, monitoring the spread of news, opinions, and scientific ideas via citation networks~\cite{adar:2004,gruhl:2004,leskovec:2005}, and detecting the spread of erroneous information~\cite{dong:2009}.

In practical applications, the underlying diffusion network (e.g. networks on who influenced whom) is often hidden. Therefore, many interesting models have been developed to automatically infer diffusion networks from cascade observations, i.e., timestamps when users post a blog on certain topics or purchase products~\cite{gomez-rodriguez:leskovec:krause:inferring,gomez-rodriguez:balduzzi:schoelkopf:uncovering,yang:zha:mutualExciting,zhou:zha:song:mutualExciting,Wang:Hu:Philip:Li:multiAspect,Daneshmand:Gomez:Song:recovery14,Du:Song:Song:Alex:HeterogeneousInf,Du:Song:Woo:Zha:topicCascade}. 


To be more specific, the diffusion network is usually modeled as a matrix $A=\{\alpha_{u,v}\}$, where $\alpha_{u,v}$ is a parameter as the strength of influence user $u$ has on $v$. Depending on the diffusion model, $\alpha_{u,v}$ can stand for the parameter of the delay distribution for information to propagate from $u$ to $v$ or the probability that $v$ publish a related post given that $u$ has published a post previously. The \emph{Network Inference} problem focuses on discovering the all parameters $\alpha_{u,v}$ from the observed cascades. 



Most previous work on network inference assume that the strength of influence $\alpha_{u,v}$ between $u$ and $v$ remains the same during the whole life of the diffusion processes. We can easily see that this assumption is unrealistic. Consider the propagation of a story in microblog as an example. It is more likely for a user to be influenced by friends to talk about the story when it is popular due to its freshness in the beginning of the propagation. On the contrary, when the story becomes stale and less popular towards the end of its life, it is less likely that one can influence her friends to have interest in it. 


Ignoring the heterogeneity of influence strength in different stages can lead to unsatisfactory solution of the Network Inference problem. Assume that we observe user $v$ posts the same contents immediately after $u$. Existing Network Inference algorithm will treat it as a strong evidence that user $u$ has strong influence on $v$. However, it is also possible that $v$ reposes fast simply because the story is very popular and any of her friends' post will lead to similar rapid response. On the other hand, if the above scenario occurs when the story is not no longer popular, $v$'s rapid response to $u$'s post can be safely treated as strong evidence that $u$ is very influential on $v$. The confusion between strong influence between friends and the popularity of the story leads to the failure in this scenario.  


In this work, we propose to incorporate the heterogeneity of influence strength in different life stage of diffusion process to improve the accuracy of network inference. We design a heuristic referred to as \emph{Life Stage Heuristics} (LSH) that can be easily applied to any existing network inference algorithm. In our LSH, we model the strength of influence as a function of the diffusion life stage $\alpha_{u,v}(t)$ instead of a constant $\alpah_{u,v}$ in previous network inference algorithms.  first approximate the popularity in different life stage by the number of activation. Then, the diffusion rate associated with each edge is  modeled as the multiplication of the popularity of the information in the current stage and the strength of influence between the pair of individuals.     

Recent work has shown that the network inference algorithms are more effective when considering the \emph{heterogeneous} influence such as topic-dependent transmission rates~\cite{Du:Song:Woo:Zha:topicCascade}, heterogenous delay distribution~\cite{Du:Song:Song:Alex:HeterogeneousInf} and multi-aspects multi-pattern cascades~\cite{Wang:Hu:Philip:Li:multiAspect}. However, most previous work ignore the heterogeneity of diffusion speed with respect to the life stage of the cascade. Instead, they assume 

 
Most previous work on network inference assumes that all propagations share the \emph{same} diffusion network. We can easily see that this assumption is unrealistic as the influence between users depends on the topic or the entity under propagation. For example, IT analysts have more influence on others regarding the choice of smart phones, while users tend to trust personal friends as a doctor on medical advices. Moreover, recent work has shown that the network inference algorithms are more effective when considering the \emph{heterogeneous} influence such as topic-dependent transmission rates~\cite{Du:Song:Woo:Zha:topicCascade}, heterogenous delay distribution~\cite{Du:Song:Song:Alex:HeterogeneousInf} and multi-aspects multi-pattern cascades~\cite{Wang:Hu:Philip:Li:multiAspect}. Inferring aspect-specific diffusion networks is a challenging problem.  Naively applying existing network inference algorithms to each aspect-specific cascades independently usually results in unsatisfactory solution. For example, for a network with a hundred nodes, they usually require at least a thousand cascades to achieve accurate inference~\cite{gomez-rodriguez:leskovec:krause:inferring}, while in practice it is unlikely that we can gather this many number of cascades for each topic. 