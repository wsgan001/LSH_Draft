Understanding the processes and dynamics of \emph{information diffusion} through networks plays a fundamental role in a variety of domains, such as evaluating the effects of networks in marketing~\cite{domingos:2001,kempe:2003,leskovec:2007a}, monitoring the spread of news, opinions, and scientific ideas via citation networks~\cite{adar:2004,gruhl:2004,leskovec:2005}, and detecting the spread of erroneous information~\cite{dong:2009}.

In practical applications, the underlying diffusion network (e.g. networks on who influenced whom) is often hidden. Therefore, many interesting models have been developed to automatically infer diffusion networks from cascade observations, i.e., timestamps when users post a blog on certain topics or purchase products~\cite{gomez-rodriguez:leskovec:krause:inferring,gomez-rodriguez:balduzzi:schoelkopf:uncovering,yang:zha:mutualExciting,zhou:zha:song:mutualExciting,Wang:Hu:Philip:Li:multiAspect,Daneshmand:Gomez:Song:recovery14,Du:Song:Song:Alex:HeterogeneousInf,Du:Song:Woo:Zha:topicCascade}. 


To be more specific, the diffusion network is usually modeled as a matrix $ISA=\{\IS_{u,v}\}$, where $\IS_{u,v}$ is a parameter as the strength of influence user $u$ has on $v$. Depending on the diffusion model, $\IS_{u,v}$ can stand for the parameter of the delay distribution for information to propagate from $u$ to $v$ or the probability that $v$ publish a related post given that $u$ has published a post previously. The \emph{Network Inference} problem focuses on discovering the all parameters $\IS_{u,v}$ from the observed cascades. 



Most previous work on network inference assume that the strength of influence $\IS_{u,v}$ between $u$ and $v$ remains the same during the whole life of the diffusion processes. We can easily see that this assumption is unrealistic. Consider the propagation of a story in microblog as an example. It is more likely for a user to be influenced by friends to talk about the story when it is popular due to its freshness in the beginning of the propagation. On the contrary, when the story becomes stale and less popular towards the end of its life, it is less likely that one can influence her friends to have interest in it. 


Ignoring the heterogeneity of influence strength in different stages can lead to unsatisfactory solution of the Network Inference problem. Assume that we observe user $v$ posts the same contents immediately after $u$. Existing Network Inference algorithm will treat it as a strong evidence that user $u$ has strong influence on $v$. However, it is also possible that $v$ reposes fast simply because the story is very popular and any of her friends' post will lead to similar rapid response. On the other hand, if the above scenario occurs when the story is not no longer popular, $v$'s rapid response to $u$'s post can be safely treated as strong evidence that $u$ is very influential on $v$. The confusion between strong influence between friends and the popularity of the story leads to the failure in this scenario.  


In this work, we incorporate the heterogeneity of influence strength in different life stage of diffusion process to improve the accuracy of network inference. We design a heuristic referred to as the \emph{Life Stage Heuristics} (LSH) that can be easily applied to any existing network inference algorithm. In our LSH, we model the strength of influence as a function of the diffusion life stage, namely $\IS_{u,v}(t)$ instead of a constant $\IS_{u,v}$ in previous network inference algorithms. We decouple the $\IS_{u,v}(t)$ as a multiplication of two component, a time varying part $\DS(t)$ modeling the popularity of information and a constant part $\IS_{u,v}$ modeling the strength of influence between pairs of individuals. Our formulation decomposes the true strength of influence from the confounding factor of popularity of the information. Moreover, the simplicity of the multiplicative form makes our heuristic easily applicable to any existing network inference algorithms ti improve inference accuracy without changes in implementation nor increment in running time.   


The rest of the paper is organized as follows. We first provide empirical evidence on the heterogeneity of influence strength and then provide a method to approximately estimate the popularity in different stages. We then show how our LSH can be applied to three sate-of-art network inference algorithms to improve the inference accuracy. We carry out extensive experiments on both synthetic and real world datasets to test the performance of our algorithms. Finally, we discuss related work and conclude the paper.  
